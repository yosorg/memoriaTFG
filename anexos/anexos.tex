\newpage
\chapter*{Apéndices}
\thispagestyle{empty}

\section{Contido DVD}

\section{Requisitos e instalación}
revisar e redactar tutorial
descargar ultima version de raspbian lite
flashear so en targeta micro sd
montar particiones creadas en la targeta
navegar a la particion boot de la sd
crear un archivo vacio llamado ssh
crear un archivo llamdo wpa_supplicant.conf
editar el archivo añadiendo la siguinte cofiguración con la cofiguracion de la red wifi creada en el movil
	country=COUNTRY
	ctrl_interface=DIR=/var/run/wpa_supplicant GROUP=netdev
	update_config=1

	network={
	       ssid="SSID"
	       psk="PASSWORD"
	       key_mgmt=WPA-PSK
	    }
#contemplar realizar configuracion con ip estatica
#añadir opcion para activar otg https://medium.com/@aallan/setting-up-a-headless-raspberry-pi-zero-3ded0b83f274
espulsamos la targeta y la introduciomos en la raspi y la encendemos
#ofrecer opciones para averiguar la ip
nos conectamos por ssh con el usuario y contraseña por defecto "pi" "raspberry"
cambiamos la contraseña por defecto con passwd
ejecutar sudo raspi-config
en el apartado Interfacacing Options activar el modulo de la camara
en el apartado advanced options ejecutar la opcion enlage filesistem
ejecutar el apartado update
salir de la configuracion
ejecutar sudo apt update
ejcutar sudo apt upgrade

instalacion de la libreia de jgarff para el control de los leds
sudo apt install scons
descargamos la libreria con wget https://github.com/jgarff/rpi_ws281x/archive/master.zip
descomprimimos con unzip master.zip
entra en la carpeta y ejecutar el comando scons
ejecutar sudo apt install python-dev swig
ejecutar wget https://pypi.python.org/packages/source/s/setuptools/setuptools-5.7.zip
entrar en la carpeta python y ejecutar sudo python ./setup.py build
ejecutar sudo python ./setup.py install

\newpage
\thispagestyle{empty}
