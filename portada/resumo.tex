%%%%%%%%%%%%%%%%%%%%%%%%%%%%%%%%%%%%%%%%%%%%%%%%%%%%%%%%%%%%%%%%%%%%%%%%%%%%%%%%

\begin{abstract}
\thispagestyle{empty}
Hoxe en día os desprazamentos en bicicleta, especialmente en cidade, están a sufrir un notable incremento. A concienciación ecoloxista e a saturación de coches que sofre o entorno urbano esta levando á xente a buscar métodos de desprazamentos unipersoais. A convivencia destes vehículos cos coches implica un alto risco de lesións en caso de accidente. Para tratar de mellorar esta situación buscarase crear un sistema para aumentar a visibilidade do usuario como a súa capacidade de visión.

Analizaremos as opcións posibles, e deseñaremos e implementaremos un sistema baseado en dous dispositivo BikeCam, un miniordenador cunha cámara e luces LED de cores que usaremos para indicar a posición e as manobras do usuario, e BikeView, unha aplicación Android que permitirá a visualización do vídeo en directo así como o control automatizado das luces.

\vspace*{25pt}
\begin{description}
\item [Palabras chave:] \mbox{} \\[-20pt]
  \begin{itemize}
    \item Seguridade
    \item Bicicleta
    \item Luces
    \item Cámara
    \item Vídeo
    \item Batería
    \item Rasberry Pi
    \item Android
  \end{itemize}
\end{description}

\end{abstract}

%%%%%%%%%%%%%%%%%%%%%%%%%%%%%%%%%%%%%%%%%%%%%%%%%%%%%%%%%%%%%%%%%%%%%%%%%%%%%%%%
