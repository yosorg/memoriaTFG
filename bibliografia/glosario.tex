\chapter{Glosario de termos}
\label{chap:glosario-termos}

%%%%%%%%%%%%%%%%%%%%%%%%%%%%%%%%%%%%%%%%%%%%%%%%%%%%%%%%%%%%%%%%%%%%%%%%%%%%%%%%
% Obxectivo: Lista de termos empregados no documento,                          %
%            xunto cos seus respectivos significados.                          %
%%%%%%%%%%%%%%%%%%%%%%%%%%%%%%%%%%%%%%%%%%%%%%%%%%%%%%%%%%%%%%%%%%%%%%%%%%%%%%%%

\begin{description}
 \item [Bytecode] Código independente da máquina que xeran
   compiladores de determinadas linguaxes (Java, Erlang,\dots) e que
   é executado polo correspondente intérprete.
 \item [Lipo]
 \item [Toolchain] Conxunto de ferramentas encadeadas para desenvolver
 un produto de software.
 \item [Buffer] Memoria utilizada temporalmente para almacenar datos de
 entrada ou saída durante a transmisión.
 \item [Kernel] Nucleo do sistema operativo, encargado da xestión do resto
 de elementos do sistema.
 \item [Aliassing] Efecto que causa que dúas sinais diferentes sexa indistinguibles
 por falta de resolución de mostraxe.
 \item [Streaming] Retransmisión en directo dun contido ou sinal.
 \item [Bitstream] Secuencia de bits.
 \item [Socket] Interface para intercambiar un fluxo de datos nunha conexión.
 \item [Bitrate] Indica o numero de bits por unidade de tempo.
 \item [Thread] Secuencia de instrucións que pode ser manexada por un planificador
 como pode ser o sistema operativo.
 \item [STL] Stereolithography, formato de arquivo para a representación
 tridimensional da estrutura dunha figura .
 \item [GCODE] Linguaxe de programación para o control numérico.
 \item [Systemd] Conxunto de daemons de administración do sistema, utilizado
  varios sistemas Linux.
 \item [Log] Rexistro escrito dun evento.
 \item [Callback] Función pasada como argumento de outra, que permite aumentar
 a abstración do código.
 \item [Script] Programa escrito nunha linguaxe de scripting, que permite a execución
 de ordes nun intérprete sen necesidade de compilación previa.

\end{description}
