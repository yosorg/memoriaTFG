\chapter{Analise e planificación}
\label{chap:Analise e planificación}
\lettrine{N}{este} capítulo partirase dos requisitos do proxecto para analizar as posibles funcionalidades planificar a implementación e elixir a solución a implantar.

\section{Análise}
O obxetivo do proxecto é a creación de dous dispositivos que realizarán as seguintes funcións:
\begin{itemize}
    \item\textbf{Informar da posición da bicicleta mediante luces}

Esta función realizarase no dispositivo principal. Para elo terá que contar con luces leds e capacidade para controlalas, tamén dispor de capacidade de comunicación co segundo dispositivo e dispoñer dunha fonte de alimentación con capacidade suficiente para facer funcionar as luces.\\

    \item\textbf{Informar das manobras e estado do vehículo mediante luces}

Para realizar esta función os requisitos son os mesmos da función anterior o que engadiremos a necesidade de sensores para detectar cambios no movemento da bicicleta, para sinalizar freadas ou accidentes, e cambios na luz do ambiente para acender as luces cando as condicións lumínicas non sexan favorables.\\

    \item\textbf{Captura de vídeo do que sucede detrás do vehículo}

Esta función tamén se realizará no dispositivo principal e necesitará os requisitos de comunicación e alimentación enerxética xa citados. Contará cunha cámara para capturar as imaxes e necesitará a capacidade para procesalas e transmitilas en tempo real.\\

    \item\textbf{Entrada de ordes do usuario para o control de luces e vídeo}

Esta función se executa no segundo dispositivo, para realizala será necesario un mecanismo de entrada, como poden ser botóns, pulsadores ou pantalla táctil. Tamén se necesitará capacidades de procesamento, comunicación e alimentación enerxética.\\

    \item\textbf{Reprodución do vídeo en tempo real}

Tamén a realizar no dispositivo dous, esta función necesita, a maiores do anteriormente citado unha pantalla na que poder ver o vídeo, e capacidades abondo para reproducilo a tempo real.
\end{itemize}

A maiores destes requisitos funcionais contase cos seguintes obxectivos:
\begin{itemize}
    \item \textbf{Pequeno tamaño e potabilidade}

Co motivo de poder dispoñer o dispositivo da bicicleta, xa sexa para cargalo ou por motivos de seguridade o deixar a bicicleta aparcada na rúa, priorizarase por un deseño portátil. O ideal e que os dispositivos poidan separase da bicicleta con facilidade e que o seu tamaño permita gardalos no peto. Este, xunto cos requisitos funcionais, é o principal motivo polo que optarase por utilizar un telefono móbil como segundo dispositivo, xa que a maioría de persoas dispoñen de un en todo momento, a así evitaríase ter que levar un dispositivo a maiores.

    \item \textbf{Independencia entre dispositivos}

Buscarase que os dispositivos poidan conectarse sen fíos para obter unha maior independencia entre os dispositivos e evitar ter que colocar cables na bicicleta. Tamén se estudará a posibilidade de utilizar unha conexión cableada para diminuír o consumo enerxético dos dispositivos, neste caso optarase preferiblemente por unha conexión USB por compatibilidade co telefono móbil.

    \item \textbf{Batería e alimentación}

Para poder alimentar o dispositivo principal necesitarase dunha batería con capacidade abonda para poder utilizalo polo menos un dia de uso sen ter que recargala. Tamén se estudará a posibilidade de incluír sistemas de aceso e apagado de ser preciso.

    \item \textbf{Sinxeleza e capacidade de actualización}

Pretendese desenvolver un sistema robusto e simple para facilitar o seu mantemento e poder actualizalo de forma sinxela. Especialmente o software ha de ser o máis simple posible para poder permitir incorporar novas funcionalidades no futuro.
\end{itemize}
\section{Planificación}

Seguindo a metodoloxía Top-Dow o desenvolvemento do proxecto dividirase en dúas fases correspondentes a os dous dispositivos. Comezarase polo desenvolvemento do dispositivo principal seguido da aplicación no dispositivo móbil é a comunicación entre ambos. Unha vez desenvolvidas as funcións principais iterarase entre as fase para desenrolar a conexión entre os dispositivos.

\subsection{Fase 1: Dispositivo principal}
As tarefas a realizar son as seguintes:
\begin{itemize}
    \item Conexión dos leds e probas de funcionamento.
    \item Funcións de control dos leds, secuencias e cores.
    \item Servidor de peticións de ordes.
    \item Conexión da cámara e probas de funcionamento.
    \item Transmisión de vídeo.
    \item Autoarranque e apagado.
    \item Alimentación enerxética e batería.
    \item Deseño e construción de carcasa e ancoraxes.
\end{itemize}
\subsection{Fase 2: Aplicación do dispositivo móbil}
Realizaranse a seguintes tarefas:
\begin{itemize}

    \item Introdución de ordes.
    \item Deseño de interfaces.
    \item Transmisión de ordes.
    \item Xestión do estado.
    \item Probas con sensores.
    \item Funcións de automatización.
    \item Recepción de vídeo.
    \item Ancoraxe do dispositivo a bicicleta.
\end{itemize}

\section{Custos do proxecto}
O software utilizado neste proxecto conta con licenzas de software libre polo que os custos restrinxiranse aos recursos humanos e recursos hardware.
Nos recursos hardware da táboa~\ref{tab:custos_hardware} inclúense as pezas utilizadas finalmente para o dispositivo asi como as que se utilizaron para realización de probas. Para aproximar o rango prezos do custo \emph{DIY} de fabricación do dispositivo  na táboa~\ref{tab:custos_dispositivo} calcularanse dúas opción a primeira utilizando os compoñentes máis baratos é a segunda os máis caros.


\begin{table}[tb]
    \label{tab:custos_hardware}
    \caption{Custos monetarios do proxecto}
    \begin{center}
        \begin{tabular}{|l|l|l|l|}
            \hline
             Compoñente & Cantidade & Custo por unidade & Subtotal\\ \hline
             Raspberry Pi Zero W & 2 & 11.00€ & 22.00€ \\ \hline
             Raspberry Pi Zero & 2 & 5.00€ & 5.00€ \\ \hline
             Caixa oficial Pi Zero & 1 & 6.00€ & 6.00€ \\ \hline
             Pi Camera Module V1 & 1 & 5.00€ & 5.00€ \\ \hline
             Pi Camera Module V2 & 2 & 20.00€ & 40.00€ \\ \hline
             Conxunto de lentes & 1 & 2.00€ & 2.00€ \\ \hline
             Tira 8 leds w2812b & 4 & 1.00€ & 4.00€\\ \hline
             Anel 8 leds w2812b & 2 & 1.00€ & 2.00€ \\ \hline
             Adafruit Powerboost 1000C & 1 & 23.00€ & 23.00€ \\ \hline
             Batería Lipo 1600mA & 1 & 7.00€ & 7.00€\\ \hline
             Batería 18650 3400mA & 2 & 2.00€ & 4.00€ \\ \hline
             Material impresión 3D PLA & <1Kg & 15.00€ & 15.00€ \\ \hline
             Cables &  &  & 5.00€ \\ \hline
             Total &  &  & 140.00€ \\ \hline
        \end{tabular}
    \end{center}
\end{table}

\begin{table}[tb]
    \label{tab:custos_dispositivo}
    \caption{Custos de fabricación do dispositivo \emph{DIY}}
    \begin{center}
        \begin{tabular}{|l|l||l|l|}
            \hline
             Menor prezo & Custo  & Maior Prezo & Custo \\ \hline
             Raspberry Pi Zero & 5.00€  & Raspberry Pi Zero W& 11.00€ \\ \hline
             Adaptador Wi-Fi usb & 1.00€ &  & \\ \hline
             Material Carcasa 3D & 0.50€ & Carcasa ofical & 6.00€ \\ \hline
              &  & Material Soprte 3D & 0.50€ \\ \hline
             Pi Camera Module V1 & 3.00€ & Pi Camera Module V2 & 20.00€ \\ \hline
              &  & Lentes  & 2.00€ \\ \hline
             8 leds w2812b & 1.00€  &  24 leds w2812b & 3.00€\\ \hline
             Circuito de carga e protección xenérico & 2.00€  & Adafruit Powerboost 1000C & 23.00€ \\ \hline
             Batería Lipo 1600mA & 7.00€ & Batería Lipo 1800mA & 10.00€\\ \hline
             Cables & 1.00€ &  Cables & 1.00€ \\ \hline
             Total & 20.50€ &  Total & 76.50€ \\ \hline
        \end{tabular}
    \end{center}
\end{table}

As estes custos engádense as ferramentas utilizadas incluíndo entre outros ordenador, impresora 3D, multímetro, soldador, dispositivo android e bicicleta.

Na táboa~\ref{tab:custos_humanos} móstranse os recursos humanos utilizados medidos en horas de traballo.

\begin{table}[tb]
    \label{tab:custos_humanos}
    \caption{Tempo empregado no proxecto}
    \begin{center}
        \begin{tabular}{|l|l|}
            \hline
             Tarefa & Tempo\\ \hline
             Documentación & 80h \\ \hline
             Desenvolvemento do dispositivo & 100h \\ \hline
             Desenvolvemento do software do dispositivo & 100h \\ \hline
             Desenvolvemento da aplicación & 200h \\ \hline
             Deseño das pezas 3D & 70h \\ \hline
             Probas e avaliación experimental & 32h\\ \hline
             Total & 512h \\ \hline
        \end{tabular}
    \end{center}
\end{table}
