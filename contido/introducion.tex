\chapter{Introdución}
\label{chap:introducion}

\lettrine{N}{este} capítulo introdutorio preséntase a motivación e os obxectivos do presente traballo.
Por último, comentarase a estrutura da memoria.

\section{Motivación}

Nos últimos anos os medios de transporte alternativos como son o caso de bicicletas, patinetes e similares están aumentando notablemente. As preocupacións medioambientais, os beneficios para a saúde e as vantaxes en termos de custo e eficiencia están a impulsar estes vehículos unipersoais. Porén, a falta de costume dos condutores xunto coas substanciais diferencias entre os vehículos e a falta de proteccións en caso de accidente dificultan a convivencia con automóbiles nas mesmas vías e implican un risco engadido para a integridade física dos condutores destes vehículos. Unha das maneiras de reducir estes riscos é aumentar a visibilidade tanto da bicicleta por parte dos coches coma viceversa co obxectivo de aumentar as distancias e os tempos de reacción dos condutores.

A maioría dos accidentes nos que a vítima é un ciclista implican un turismo na colisión~\cite{PrincipalesCifrasSiniestralidad} .
O estudo da Universidade de Valencia~\cite{ESTUDIOANALISISSINIESTRALIDAD} con datos da Dirección Xeneral de Tráfico mostra que a pesar de que a maioría de accidentes ciclistas prodúcense de día e con boas condicións de visibilidade, momento no que hay maior número desprazamentos en bicicleta, no caso de accidentes con pouca luminosidade a gravidade das lesións é superior como se mostra na táboa~\ref{tab:lesividade}.


\begin{table}[tb]
  \caption{Lesividade en función da luminosidade da vía~\cite{ESTUDIOANALISISSINIESTRALIDAD}. }
  \label{tab:lesividade}
    \begin{center}
        \begin{tabular}{|l|l|l|l|l|}
            \hline
             & Morto & Ferido grave & Ferido leve & Total\\ \hline
             Pleno día & 1.1\(\%\)& 13.7\(\%\) & 85.1\(\%\)& 100\(\%\)\\ \hline
             Crepúsculo & 1.9\(\%\) & 13.0\(\%\) &85.1\(\%\) &100\(\%\)\\ \hline
             Iluminación suficiente (noite) &0.7\(\%\) &9.3\(\%\) &90.0\(\%\)  & 100\(\%\)\\ \hline
             Iluminación insuficiente (noite)  & 2.8\(\%\)& 17.7\(\%\)& 79.5\(\%\)& 100\(\%\)\\ \hline
             Sen iluminación (noite) & 10.3\(\%\)& 27.4\(\%\)&62.4\(\%\) & 100\(\%\) \\ \hline
        \end{tabular}
    \end{center}

\end{table}

Varios estudos mostran que o uso de luces en bicicletas, especialmente as dinámicas permiten que estas sexan divisadas a maiores distancias tanto de día como de noite. A tese The Nighttime Conspicuity Benefits of Static and Dynamic Bicycle Taillights~\cite{edewaardNighttimeConspicuityBenefits2017} estuda os beneficios de diferentes luces traseiras de bicicleta de noite. Conclúe que as luces que se moven co ciclista, como as colocadas nos nocellos, son as que máis visibilidade aportan pero cando o ciclista  deixa de pedalear estes beneficios se perden polo que o uso dunha luz fixa cun patrón de palpadeo pode ser a mellor opción en tódalas circunstancias.

Por outra parte a falta de retrovisores na maioría de bicicletas implica que o ciclista debe xirase cada vez que quere saber o que está a acontecer tras el facendo que así perda momentaneamente a visión do que ocorre diante e incluso o equilibrio en ciclistas non experimentados.

Para paliar estes problemas exponse unha solución baseada nun ou varios dispositivos dotados de luces e cámara xunto a outro de control e visualización.


\section{Obxectivos}
\label{sec:obxectivos}
Os obxectivos principais de este proxecto serán dous:
O primeiro de desenvolvemento de dous dispositivos diferenciados, un dotado de cámara e luces que se poderá colocar en diferentes lugares da bicicleta e do que se poderán utilizar unha ou varias unidades o mesmo tempo que disporá de alimentación propia ou compartida. Un segundo dispositivo de interacción co usuario para o manexo do primeiro e a visualización do vídeo capturado. Tamén será necesario o hardware que permita a comunicación dos dispositivos xa sexa por cable ou sen fíos.

O segundo consistirá en desenvolver o software que permita o funcionamento dos dispositivos. O control das luces para indicar posición aumentar a visibilidade ou indicar manobras coma a freada ou xiro. O control do vídeo permitindo activalo e desactivalo cando se desexe. As posibles automatizacións como o acendido de luces cando haia pouca visibilidade ou a indicación automática da freada. As interfaces de iteración co usuario que permitan o control e a visualización sen distraccións da condución. A integración e comunicación entre os dispositivos en tempo real e de forma transparente para o usuario.

Se comenzará coa análise das posíbeis solucións contemplando as diferentes opcións de hardware dispoñibles tendo en conta o custo, o tamaño, as capacidades de funcionamento, as restricións de compatibilidade, as restricións no software a utilizar, a dispoñibilidade e a dificultade de uso polo usuario final.

Se realizará a implementación da solución elixida e se someterá a probas nun entorno real para comprobar o cumprimento dos requisitos establecidos.


\section{Proposta}
Neste traballo propoñerase unha solución baseada nun microordenador colocado baixo a sela, a Raspberry Pi Zero, que contará cunha cámara é unhas serie de luces leds conectadas. Para o control e a visualización do vídeo en directo realizarase unha aplicación Android que se executará nun teléfono situado no guiador da bicicleta.


\section{Traballo relacionado}
Existen no mercado dispositivos con funcións similares pero poucos integra tódalas funcións os dous máis avanzados son os seguintes.

 A cámara Fly6 da compañía Cycliq~\cite{Fly6CERear}, un combo de cámara traseira máis luces que se poden controlar dende o móbil, pero non permite o \emph{streaming} en directo do vídeo, solo a gravación. Conta con características interesantes como a estabilización da imaxe, resistencia a auga e un tamaño moi compacto ou seu prezo e de 179 euros.

 A outra opción existente é a cámara Hexagon~\cite{HEXAGONCameraSignals} que se financiou exitosamente mediante croudfounding no ano 2017 pero nunca chegou a produción. Este dispositivo si que contaría con \emph{streaming} de vídeo en directo, xunto con acendido automático das luces en caso de freada. A aplicación tamén se encarga de gravar a posición gps no itinerario ou compartila en tempo real, conta cun segundo dispositivo con botóns para o control do dispositivo, e de chegar a producirse o seu prezo estaría entre os 100 e 200 euros.

 A principal diferenza destes dispositivos con este proxecto é o uso dunha arquitectura aberta e modular que non so integra máis funcionalidades se non que sentará como base para a implantación de moitas máis no futuro.


\section{Estrutura da memoria}
Este documento estrutúrase en seis capítulos e un anexo.
\begin{itemize}
  \item Neste primeiro capítulo exponse a motivación os requisitos e as liñas xerais do proxecto, incluíndo os obxectivos e finalmente faise unha proposta e compárase con traballos relacionados.
  \item No capitulo 2 expóñense as funcionalidades requiridas para o sistema, faise elíxese a metodoloxía a utilizar e partindo de esta planifícanse as etapas de desenvolvemento e por ultimo móstranse os custo do proxecto.
  \item O capítulo 3 presenta a arquitectura do sistema e explora as posibles alternativas dispoñibles analizando as vantaxes e desvantaxes de cada unha delas fronte a solución elixida.
  \item O capítulo 4 relata o proceso, os detalles e os problemas xurdidos na implementación e construción do dispositivo BikeCam xunto co seu software.
  \item O capítulo 5 expón o deseño e implementación de BikeView a aplicación Android que se utilizará para o control e a visualización.
  \item O capítulo 6 describe as probas realizadas e a análise dos seus resultados. O sexto e derradeiro capítulo recolle as conclusión obtidas trala realización deste proxecto.
  \item O capítulo 7 é o derradeiro no que se presentan as conclusións finais do proxecto valorando o proceso levado a cabo e os resultados obtidos, finalmente se formulan opcións de traballo futuro e posibles melloras do proxecto.
\end{itemize}
