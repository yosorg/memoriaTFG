correspóndense\chapter{Evaluación experimental}
\label{chap:evaluacion_experimenal}
Para comprobar que os resultados obtidos correspóndense co plantexado probaremos o dispositivo e a aplicación. Realizaremos probas en contornos controlados e probas no medio real o que esta dirixido.

\section{Consumo e autonomía}

Comenzarase as probas medindo o consumo de amperios do sistema. Para elo colocarase un amperímetro usb entre unha fonte de alimentación e a conexión usb coa que alimentarase o sistema nestas probas. Repetiremos as probas con diferentes fontes de alimentación e distintos cables para descartar fallos e conseguir unha maior consistencia nos resultados. A continuación realizaranse as mesmas probas no dispositivo medindo o consumo alimentándoo coa batería.

Analizaranse os seguintes supostos para o dispositivo con un anel led e para o que conta a maiores coas dúas tiras led.
\begin{itemize}
    \item \textbf{Sistema en repouso: }
    O sistema esta aceso pero só se estean a executar as funcións do sistema operativo incluíndo o servidor ssh para o control remoto.
    \item \textbf{Servidor funcionando:}
    Executamos o servidor.
    \item \textbf{Cliente conectado:}
    Conectamos o dispositivo móbil o servidor.
    \item \textbf{Vídeo transmitindo:}
    Transmitimos vídeo en directo o dispositivo móbil.
    \item \textbf{Vídeo parado:}
    Paramos a transmisión de vídeo.
    \item \textbf{Desconexión do cliente:}
    Pechamos a aplicación no dispositivo móbil.
    \item \textbf{Luces intermitentes:}
    Iniciamos as luces intermitentes a máxima intensidade nunha das direccións, consumo varia no proceso, rexistremos o valor máximo.
    \item \textbf{Luces vermellas:}
    Acendemos as luces vermellas a intensidade máxima.
    \item \textbf{Luces vermellas e transmisión de vídeo:}
    Consumo coas luces vermellas a máxima intensidade e co vídeo transmitindo en directo.
\end{itemize}

Os resultados obtidos son os seguintes:

\begin{table}[tb]
    \label{c:comparativa}
    \begin{center}
        \begin{tabular}{|l|l|l|l|}
            \hline
             &  24 luces & 24 luces & 8 luces\\
             & & e batería & \\ \hline
             Sistema en repouso & 165mA & 148mA & \\ \hline
             Servidor en  funcionamento & 165mA & 148mA & \\ \hline
             Cliente conectado & 165/168mA  & 148mA & \\ \hline
             Vídeo transmitindo & 360mA & 342mA & \\ \hline
             Vídeo parado & 211mA & 191mA & \\ \hline
             Desconexión do cliente & 165mA & 148mA & \\ \hline
             Luces intermitentes & 428mA & 414mA & \\ \hline
             Luces vermellas & 505mA & 502mA & \\ \hline
             Luces vermellas e vídeo & 660mA & 645mA & \\ \hline
        \end{tabular}
    \end{center}
    \caption{Comparativa de consumo de amperios}
    \label{tab:my_label}
\end{table}


As fontes de alimentación utilizadas poden proporcionar un máximo de 3A e 2.1A respectivamente, tamén utilizáronse dous cables un de maior calidade e outro cunha calidade inferior. En ningún dos casos obtivéronse diferencias apreciables sendo a maior diferencia de 4mA. Na versión con batería necesitouse utilizar cableado a maiores para realizar as probas, o que pode que incrementara lixeiramente o consumo.

A seguinte proba a realizar será a de autonomía do dispositivo, para elo buscarase o consumo máximo acendendo as luces vermella a máxima intensidade o tempo que se transmite vídeo.

Comezaranse as probas co dispositivo dotado de batería interna, cando a voltaxe da batería esta entorno os 3.7V detectase un problema o sistema apagase, indicando batería baixa. O \emph{script} Python encargado de monitorizar o pin conectado o indicador de voltaxe baixo do Adafruit Powerboost detecta unha caída de voltaxe que confunde co franco de caída esperado cando o pin ponse a 0. O pin de baixo voltaxe do Adafruit Powerboost está conectado a voltaxe da batería cando a caga e superior a 3.2V e conectase a un valor de 0V cando baixa de este limite e debido a que cando batería esta a comezar a descargarse para poder proporcionar a amperaxe adecuada a súa voltaxe baixa a un ritmo rápido que confunde o \emph{script}. O podemos solucionar incluíndo unha segunda comprobación no \emph{script} despois de detectar o franco de caída comprobarase que o valor lido no pin é 0 antes de apagar a Raspberry.

A continuación detectase un segundo problema, despois de  min volvese a apagar por batería baixa e unha vez apagado a batería recuperase ata unha voltaxe de 3.5V. Esto debese a que o estar consumindo o máximo de amperios necesita baixar a súa voltaxe para poder seguir entregando estes amperios. Que o sistema se apague neste punto e conveniente para protexer a batería e prolongar a súa vida útil pero reduce o tempo de uso da batería, que poderíamos seguir usando sempre que non utilicemos o consumo máximo do dispositivo. Este problema pódese solucionar de dúas formas:
Utilizando un batería de maior capacidade xa que poderá seguir subministrando a amperaxe necesaria a menor voltaxe.
Utilizar unha batería cunha constante de descarga maior, isto é a capacidade máxima de amperios que pode subministrar en función da capacidade en amperios hora. A batería utilizada ten unha constante de descarga de 1C isto quere dicir que coa sua caga completa o fabricante garante que proporcione 1.6A funcionando con normalidade.
Ámbalas dúas solucións implican na practica unha batería de maior tamaño.

Procedemos a repetir a proba pero esta vez deshabitando o apagado automático, esperando a que o sistema se apague cando a voltaxe non sexa suficiente para manter a Raspberry acesa ou no peor caso cando o circuíto de protección que inclúe a batería a desconecte a 2.5V. De non usar unha batería con circuíto de protección poderíamos danar a batería podendo incluso arder ou estoupar durante próximas cargas. Neste caso obtemos minutos de funcionamento.

Comezarase as probas para a versión do dispositivo sen luces intermitentes, faremos a proba cunha batería de 6000mA, obtense    , Neste caso observase que cando o dispositivo non pode subministrar suficiente voltaxe para alimentar a Raspberry esta se apaga, e como consecuencia tamén os leds alimentados dende o seu \emph{power rail}, o diminuír o consumo a voltaxe volve a ser suficiente para alimentar a Rasberry polo que esta se reinicia, entrando desta forma nun ciclo de reinicios ata que a voltaxe non e suficiente para arrancar a Raspberry.

\section{Vídeo e lentes}
Nesta sección procederase a analizar a calidade do vídeo obtida a latencia
\section{Visibilidade}

\section{Estabilidade e consumo da aplicación}
